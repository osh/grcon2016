%%%%%%%% GRCON 2016 EXAMPLE LATEX SUBMISSION FILE %%%%%%%%%%%%%%%%%

\documentclass{article}

\usepackage{times}
\usepackage{graphicx} % more modern
\usepackage{subfigure} 
\usepackage{natbib}
\usepackage{algorithm}
\usepackage{algorithmic}
\usepackage{hyperref}

\newcommand{\theHalgorithm}{\arabic{algorithm}}

% please use this for your draft submission
\usepackage{grcon2016} 
% please use this for your camera ready paper
%\usepackage[accepted]{grcon2016}

\grcontitlerunning{Submission and Formatting Instructions for GRCON 2016}

\begin{document} 

\twocolumn[
\grcontitle{Submission and Formatting Instructions for \\
           GNU Radio Conference (GRCON 2016)}

\grconauthor{Your Name}{email@yourdomain.edu}
\grconaddress{Your Fantastic Institute,
            14142 Imaginary Way, Mountain View, CA 94043 USA}
\grconauthor{Your CoAuthor's Name}{email@coauthordomain.edu}
\grconaddress{Their Fantastic Institute,
            1571 Jay St., Arlington, VA 22203 USA}

\grconkeywords{software radio, gnu radio, dsp, GRCON}

\vskip 0.3in


\begin{abstract} 
The purpose of this document is to provide both the basic paper template and submission guidelines. 
Abstracts should be a single paragraph, between 4--6 sentences long, ideally. 
\end{abstract} 

\section{Electronic Submission}
\label{submission}

Submission to GRCON 2016 will be entirely electronic.  
Information about the submission process and \LaTeX\ templates
are available on the conference web site at:

\begin{center}
\url{http://gnuradio.org/grcon-2016/papers/}
\end{center}

Questions and submission can be sent to 
\texttt{grcon@gnuradio.org}.

The guidelines below will be enforced for initial submissions and
camera-ready copies.  Here is a brief summary:
\begin{itemize}
\item Submissions must be in PDF.
\item The maximum paper length is \textbf{8 pages excluding references and acknowledgements, and 10 pages
  including references and acknowledgements} (pages 9 and 10 must contain only references and acknowledgements).

\item Your paper should be in \textbf{10 point Times font}.
\item Make sure your PDF file only uses Type-1 fonts.
\item Place figure captions {\em under} the figure (and omit titles from inside
the graphic file itself).  Place table captions {\em over} the table.
\item References must include page numbers whenever possible and be as complete
as possible.  Place multiple citations in chronological order.  
\item Do not alter the style template; in particular, do not compress the paper
format by reducing the vertical spaces.
\item Keep your abstract brief and self-contained, one
   paragraph and roughly 4--6 sentences.  Gross violations will require correction at the camera-ready phase.
  Title should have content words capitalized.
\item Where possible, provide source code and example implementation via reference to your github account.
  

\end{itemize}

\subsection{Submitting Papers}

{\bf Paper Deadline:} The deadline for paper submission to GRCON 2016
is at \textbf{23:59 Universal Time (3:59 p.m.\ Pacific Standard Time) on February July 1, 2016}.
If your full submission does not reach us by this time, it may not be 
considered for publication.

{\bf Simultaneous Submission:} GRCON will not accept any paper which,
at the time of submission, is under review for another conference or
has already been published. This policy also applies to papers that
overlap substantially in technical content with conference papers
under review or previously published. GRCON submissions must not be
submitted to other conferences during GRCON's review period. Authors
may submit to GRCON substantially different versions of journal papers
that are currently under review by the journal, but not yet accepted
at the time of submission. Informal publications, such as technical
reports or papers in workshop proceedings which do not appear in
print, do not fall under these restrictions.

\medskip

To ensure our ability to print submissions, authors must provide their
manuscripts in \textbf{PDF} format.  Furthermore, please make sure
that files contain only Type-1 fonts (e.g.,~using the program {\tt
  pdffonts} in linux or using File/DocumentProperties/Fonts in
Acrobat).  Other fonts (like Type-3) might come from graphics files
imported into the document.

Authors using \textbf{Word} must convert their document to PDF.  Most
of the latest versions of Word have the facility to do this
automatically.  Submissions will not be accepted in Word format or any
format other than PDF. 

\subsection{Feedback from Paper Submission}

Acceptance notification will occur within one week of paper submission via
responding to the email address from which your PDF was submitted.

\subsection{Submitting Final Camera-Ready Copy}

The final versions of papers accepted for publication should follow the
same format as initial submissions, but should use the accepted keyword 
for the style package.

Simply change:
$\mathtt{\backslash usepackage\{grcon2016\}}$ to 

$$\mathtt{\backslash usepackage[accepted]\{grcon2016\}}$$

\noindent

Camera-ready copies should have the title of the paper as running head
on each page except the first one.  The running title consists of a
single line centered above a horizontal rule which is $1$ point thick.
The running head should be centered, bold and in $9$ point type.  The
rule should be $10$ points above the main text.  For those using the
\textbf{\LaTeX} style file, the original title is automatically set as running
head using the {\tt fancyhdr} package which is included in the GRCON
2016 style file package.  In case that the original title exceeds the
size restrictions, a shorter form can be supplied by using

\verb|\grcontitlerunning{...}|

just before $\mathtt{\backslash begin\{document\}}$.
Authors using \textbf{Word} must edit the header of the document themselves.

\section{Format of the Paper} 
 
All submissions must follow the same format to ensure the printer can
reproduce them without problems and to let readers more easily find
the information that they desire.

\subsection{Length and Dimensions}

Papers must not exceed eight (8) pages, including all figures, tables,
and appendices, but excluding references and acknowledgements. When references and acknowledgements are included,
the paper must not exceed ten (10) pages.
Acknowledgements should be limited to grants and people who contributed to the paper.
Any submission that exceeds 
this page limit or that diverges significantly from the format specified 
herein will be rejected without review.

The text of the paper should be formatted in two columns, with an
overall width of 6.75 inches, height of 9.0 inches, and 0.25 inches
between the columns. The left margin should be 0.75 inches and the top
margin 1.0 inch (2.54~cm). The right and bottom margins will depend on
whether you print on US letter or A4 paper, but all final versions
must be produced for US letter size.

The paper body should be set in 10~point type with a vertical spacing
of 11~points. Please use Times typeface throughout the text.

\subsection{Title}

The paper title should be set in 14~point bold type and centered
between two horizontal rules that are 1~point thick, with 1.0~inch
between the top rule and the top edge of the page. Capitalize the
first letter of content words and put the rest of the title in lower
case.

\subsection{Author Information for Submission}
\label{author info}

Author information should include full names, institutional affiliation,
and at least city address, as well as email or web addresses.

\subsection{Abstract}

The paper abstract should begin in the left column, 0.4~inches below
the final address. The heading `Abstract' should be centered, bold,
and in 11~point type. The abstract body should use 10~point type, with
a vertical spacing of 11~points, and should be indented 0.25~inches
more than normal on left-hand and right-hand margins. Insert
0.4~inches of blank space after the body. Keep your abstract brief and 
self-contained,
limiting it to one paragraph and roughly 4--6 sentences.  Gross violations will require correction at the camera-ready phase.

\subsection{Partitioning the Text} 

You should organize your paper into sections and paragraphs to help
readers place a structure on the material and understand its
contributions.

\subsubsection{Sections and Subsections}

Section headings should be numbered, flush left, and set in 11~pt bold
type with the content words capitalized. Leave 0.25~inches of space
before the heading and 0.15~inches after the heading.

Similarly, subsection headings should be numbered, flush left, and set
in 10~pt bold type with the content words capitalized. Leave
0.2~inches of space before the heading and 0.13~inches afterward.

Finally, subsubsection headings should be numbered, flush left, and
set in 10~pt small caps with the content words capitalized. Leave
0.18~inches of space before the heading and 0.1~inches after the
heading. 

Please use no more than three levels of headings.

\subsubsection{Paragraphs and Footnotes}

Within each section or subsection, you should further partition the
paper into paragraphs. Do not indent the first line of a given
paragraph, but insert a blank line between succeeding ones.
 
You can use footnotes\footnote{For the sake of readability, footnotes
should be complete sentences.} to provide readers with additional
information about a topic without interrupting the flow of the paper. 
Indicate footnotes with a number in the text where the point is most
relevant. Place the footnote in 9~point type at the bottom of the
column in which it appears. Precede the first footnote in a column
with a horizontal rule of 0.8~inches.\footnote{Multiple footnotes can
appear in each column, in the same order as they appear in the text,
but spread them across columns and pages if possible.}

\begin{figure}[ht]
\vskip 0.2in
\begin{center}
\centerline{\includegraphics[width=\columnwidth]{gr.png}}
\caption{GNU Radio Logo}
\label{figure:gr-logo}
\end{center}
\vskip -0.2in
\end{figure} 

\subsection{Figures}
 
You may want to include figures in the paper to help readers visualize
your approach and your results. Such artwork should be centered,
legible, and separated from the text. Lines should be dark and at
least 0.5~points thick for purposes of reproduction, and text should
not appear on a gray background.

Label all distinct components of each figure. If the figure takes the
form of a graph, then give a name for each axis and include a legend
that briefly describes each curve. Do not include a title inside the
figure; instead, the caption should serve this function.

Number figures sequentially, placing the figure number and caption
{\it after\/} the graphics, with at least 0.1~inches of space before
the caption and 0.1~inches after it, as in
Figure~\ref{figure:gr-logo}.  The figure caption should be set in
9~point type and centered unless it runs two or more lines, in which
case it should be flush left.  You may float figures to the top or
bottom of a column, and you may set wide figures across both columns
(use the environment {\tt figure*} in \LaTeX), but always place
two-column figures at the top or bottom of the page.


\subsection{Citations and References} 

Please use APA reference format regardless of your formatter
or word processor. If you rely on the \LaTeX\/ bibliographic 
facility, use {\tt natbib.sty} and {\tt grcon2016.bst} 
included in the style-file package to obtain this format.

This template is based off the excellent template for ICML provided
in the work by \cite{langley00}.

\subsection{Software and Data}

We strongly encourage the publication of software and data with the
camera-ready version of the paper whenever appropriate.  This can be
done by including a URL in the camera-ready copy.  

% In the unusual situation where you want a paper to appear in the
% references without citing it in the main text, use \nocite


\bibliography{example_paper}
\bibliographystyle{grcon2016}

\thebibliography

\end{document} 
